\documentclass[10pt]{article}

% $Date: 2011-08-17 12:45:45 +0100 (Wed, 17 Aug 2011) $
% $Revision: 47 $

\newcommand{\longtitle}{ch\_xcorr.m} % title of the document (for main title)
\newcommand{\shorttitle}{ch\_xcorr.m} % title of the document (for header)
\newcommand{\me}{Chris Hummersone} % author name
\newcommand{\email}{christopher.hummersone@surrey.ac.uk} % author email address
\newcommand{\mydate}{\today} % the date

% $Date: 2011-08-17 12:45:45 +0100 (Wed, 17 Aug 2011) $
% $Revision: 47 $

%%%%%%%%%%%%%%% GEOMETRY %%%%%%%%%%%%%%%%

% page size and margins

\usepackage[lmargin=3cm,rmargin=3cm,tmargin=3cm,bmargin=3cm]{geometry}
\geometry{a4paper}

%%%%%%%%%%%%%% TITLESEC %%%%%%%%%%%%%%%%

% section headings format

\usepackage[sf,medium,compact]{titlesec}
\titlelabel{\thetitle.\quad}

%%%%%%%%%%%%%%%% MATHS %%%%%%%%%%%%%%%%

\usepackage{amsmath,amsbsy,amssymb} % packages

\allowdisplaybreaks % allow equations to break across pages

% symbols
\newcommand{\R}{\mathfrak{R}}
\newcommand{\Z}{\mathbb{Z}}
\newcommand{\N}{\mathbb{N}}
\newcommand{\F}{\mathcal{F}}
\newcommand{\ud}{\,\text{d}}
\newcommand{\fs}{f\negthinspace{}s}
\DeclareMathOperator*{\argmax}{arg\,max}
\DeclareMathOperator*{\argmin}{arg\,min}

% commands
\newcommand{\tf}[1]{\mathbf{#1}}
\newcommand{\tfg}[1]{\boldsymbol{#1}}
\newcommand{\norm}[1]{\hat{#1}}
\newcommand{\new}[1]{\acute{#1}}
\newcommand{\peak}[1]{\mathring{#1}}
\newcommand{\pool}[1]{\bar{#1}}
\providecommand{\abs}[1]{\lvert#1\rvert}
\newcommand{\CI}{i} % channel index
\newcommand{\FI}{l} % frame index

%%%%%%%%%%%%%%%% FANCYHDR %%%%%%%%%%%%%%%%

% headers and footers

\usepackage{fancyhdr}
\fancyhead{} % clear all header fields
\fancyhead[R]{\sf\shorttitle}
\fancyfoot{} % clear all footer fields
\fancyfoot[R]{\sf\thepage}
\renewcommand{\headrulewidth}{0.5pt}
\renewcommand{\footrulewidth}{0.5pt}
\fancypagestyle{plain}{\fancyhead{} % get rid of headers on plain pages
\renewcommand{\headrulewidth}{0pt}} % and the line
\setlength{\headheight}{15pt}

%%%%%%%%%%%%%%% CAPTION %%%%%%%%%%%%%%%%

% captions and subcaptions

\usepackage[margin=2em,singlelinecheck=on,font=small,labelfont={bf,sf},labelsep=colon]{caption}
\usepackage[singlelinecheck=on,list=true]{subcaption}

%%%%%%%%%%%%%%% TOCLOFT %%%%%%%%%%%%%%%%

% TOC, LOF, and LOT

\usepackage[titles]{tocloft}
\renewcommand{\cftsecdotsep}{\cftdotsep}

%%%%%%%%%%%%%%%% TABLES %%%%%%%%%%%%%%%%

\usepackage{multirow} % allows entries to span multiple rows (identical to multicolumn command)
\newcommand{\colheading}[1]{\multirow{2}{*}{\textbf{#1}}} % custom column heading command
\newcommand{\tablesubtitle}[2]{\multicolumn{#1}{l}{\emph{\textbf{#2}}}} % custom table sub heading to span the specified number of columns
\renewcommand\arraystretch{1.2} % row height

%%%%%%%%%%%%%%%% DEFINITIONS %%%%%%%%%%%%%%%%

% words containing special symbols

\newcommand{\RT}{RT$_{\text{60}}$}
\newcommand{\matlab}{\textsc{matlab}}

%%%%%%%%%%%%%%%% MCODE %%%%%%%%%%%%%%%%

% for publishing Matlab code

\usepackage[numbered,framed]{mcode}
\lstset{breaklines=true}

\lstset{
    literate={~}{$\sim$}{1} % set tilde as a literal (no process)
}

% New list environment
\newenvironment{mlist}[1][\quad]%
  {\begin{list}{}{%
     \renewcommand{\makelabel}[1]{\small\texttt{##1}\hfil}%
     \setlength{\labelwidth}{2em}%
     \setlength{\itemsep}{0in}%
     \setlength{\parsep}{0in}%
     \setlength{\leftmargin}{\labelwidth+\labelsep}}}
  {\end{list}}

%%%%%%%%%%%%%%%% MISC %%%%%%%%%%%%%%%%

% miscellaneous packages

\usepackage[T1]{fontenc} % font package
\usepackage{textcomp} % misc symbols
\usepackage{verbdef}
\usepackage{calc} % perform calculations on lengths
\usepackage{appendix}
\usepackage{graphicx}
\renewcommand{\ttdefault}{pcr} % default typewriter font

%%%%%%%%%%%%%%%% HYPERREF %%%%%%%%%%%%%%%%

% package for adding enhanced PDF functionality (clickable links, metadata, etc.)

% Hyperref package
\usepackage[pdftex]{hyperref} 
\hypersetup{plainpages=false,%
pdftitle=\longtitle,%
pdfauthor=\me,%
colorlinks=false,
breaklinks=true
}

\renewcommand{\equationautorefname}{Equation}
\renewcommand{\figureautorefname}{Figure}
\renewcommand{\subfigureautorefname}{Figure}
\renewcommand{\Itemautorefname}{Item}
\renewcommand{\tableautorefname}{Table}
\renewcommand{\sectionautorefname}{Section}
\renewcommand{\subsectionautorefname}{Section}
\renewcommand{\subsubsectionautorefname}{Section}

% redefine appendix autoref in article class
\makeatletter
\edef\Hy@chapterstring{\expandafter\strip@prefix\meaning\Hy@chapterstring}
\makeatother 

%%%%%%%%%%%%%%%% LAYOUT OPTIONS %%%%%%%%%%%%%%%%

\hyphenpenalty=5000
\tolerance=1000
\widowpenalty=10000
\clubpenalty=10000
\raggedbottom
\setlength{\parindent}{0pt} % set paragraph indent
\setlength{\parskip}{1ex plus 0.2ex minus 0.2ex} % set paragraph spacing
\pagestyle{fancy}

%%%%%%%%%%%%%%%% DOCUMENT PROPERTIES %%%%%%%%%%%%%%%%

\title{\longtitle}
\author{\me\footnote{\href{mailto:\email}{\email}}}
\date{\mydate}
 % input standard preamble

%%%%%%%%% Begin Document %%%%%%%%%

\begin{document}

{\sf\maketitle}

This document briefly explains the workings of the \mcode{ch_xcorr} algorithm.

\pdfbookmark{Algorithm}{a}\section*{Algorithm}


\begin{lstlisting}
function [ccg,ic] = ch_xcorr(hc_L,hc_R,frameCount,frame_length,noverlap,maxlag,tau,...
    varargin)
\end{lstlisting}

\begin{lstlisting}
% SYNTAX
%   
%   CCG = CH_XCORR(HC_L,HC_R,FRAME_LENGTH,NOVERLAP,MAXLAG,TAU)
%   CCG = CH_XCORR(...,INHIB)
%   CCG = CH_XCORR(...,IC_T,NORM_FLAG)
%   CCG = CH_XCORR(...,INHIB_MODE)
%   [CCG,IC] = CH_XCORR(...)
\end{lstlisting}

For the purposes of this document:\\*
\mcode{HC_L} = $\tf{h}_L$\\*
\mcode{HC_R} = $\tf{h}_R$\\*
\mcode{FRAME_LENGTH} = $M$\\*
\mcode{NOVERLAP} = $N$\\*
\mcode{TAU} = $\tau$\\*
\mcode{INHIB} = $\iota$\\*
\mcode{IC_T} = $\Theta_\chi$\\*
\mcode{FRAME_COUNT = floor((max(size(HC_L))-MAXLAG-1)/(FRAME_LENGTH))-NOVERLAP+1} = $L$\\*

The algorithm takes two matrices of data \mcode{hc_L} and \mcode{hc_R} (such as the output of a peripheral ear model) and divides it into frames of length \mcode{frame_length} (in samples). From these data, cross-correlograms (averaged cross-correlations) are calculated for each frame. The cross-correlations and cross-correlograms can be calculated in numerous ways, with or without normalisation, and with optional inhibition.

Specifically, cross-correlations are calculated in frequency channel index $\CI$, sample index $n$ and lag index $m$ using the input data in the following way:
\begin{equation}
\norm{\tf{c}}(\CI,m,n) = \frac{\new{\tf{c}}(\CI,m,n)}{\sqrt{\tf{a}_{L}(\CI,m,n)\tf{a}_{R}(\CI,m,n)}}
\end{equation}
where
\begin{gather}
\new{\tf{c}}(\CI,m,n) = \frac{1}{\tau} \tf{h}_L\bigl(\CI,\max(n+m,n)\bigr)\tf{h}_R\bigl(\CI,\max(n-m,n)\bigr) + \biggl(1-\frac{1}{\tau}\biggr)\new{\tf{c}}(\CI,m,n-1),\\
\tf{a}_{L}(\CI,m,n) = \frac{1}{\tau} \tf{h}_L^2\bigl(\CI,\max(n+m,n)\bigr) + \biggl(1-\frac{1}{\tau}\biggr)\tf{a}_{L}(\CI,m,n-1),\\\tf{a}_{R}(\CI,m,n) = \frac{1}{\tau} \tf{h}_R^2\bigl(\CI,\max(n-m,n)\bigr) + \biggl(1-\frac{1}{\tau}\biggr)\tf{a}_{R}(\CI,m,n-1)\;
\end{gather}
and $\tau$ is the time constant of the exponentially decaying window in samples. This is used to extract the Interaural Coherence (IC) $\chi$ as:
\begin{equation}
\chi(\CI,n) = \max_{m}\norm{\tf{c}}(\CI,m,n)
\end{equation}
However, the data used in subsequent stages of the algorithm depends on \mcode{norm_flag}. Subsequent stages may use the normalised cross-correlation $\norm{\tf{c}}$ (if \mcode{norm_flag} = 1), or the un-normalised cross-correlation $\new{\tf{c}}$ (if \mcode{norm_flag} $\neq$ 1).

Following this, inhibition is applied, although this procedure depends upon the \mcode{inhib_mode} supplied to the function. The default (\mcode{'multiply'}) is a multiplication mechanism:
\begin{equation}
\tf{c}(\CI,m,n) = \iota(\CI,n)\new{\tf{c}}(\CI,m,n) \qquad (\text{\mcode{norm_flag}} \neq 1)
\end{equation}
or
\begin{equation}
\tf{c}(\CI,m,n) = \iota(\CI,n)\norm{\tf{c}}(\CI,m,n) \qquad (\text{\mcode{norm_flag}} = 1)
\end{equation}
Alternatively, a subtractive procedure (\mcode{'subtract'}) may be used:
\begin{equation}
\tf{c}(\CI,m,n) = \max\biggl(\new{\tf{c}}(\CI,m,n)-\frac{1}{\tau}\iota(\CI,n),0\biggr) \qquad (\text{\mcode{norm_flag}} \neq 1)
\end{equation}
or
\begin{equation}
\tf{c}(\CI,m,n) = \max\biggl(\norm{\tf{c}}(\CI,m,n)-\frac{1}{\tau}\iota(\CI,n),0\biggr) \qquad (\text{\mcode{norm_flag}} = 1)
\end{equation}
If no inhibition is specified then $\iota(\CI,n) = 1\ (\forall\ \CI,n)$.

The cross-correlograms $\tf{C}$ for frame $\FI$ are calculated by averaging only the inhibited cross-correlations within a given frame for which the corresponding IC value exceeds a threshold value $\Theta_{\chi}$:
\begin{equation}
\tf{C}(\CI,\FI,m) = \begin{cases}
0 \quad &\text{if } \Psi=\varnothing\\
\displaystyle\frac{1}{\abs{\Psi}}\sum_{d\in\Psi} \tf{c}(\CI,d,m) \quad &\text{otherwise}
\end{cases}
\end{equation}
where $\{\Psi\in n : (\FI-1)M+1\leq n\leq \FI M,l\leq L-N+1,\tfg{\chi}(\CI,n) \geq \Theta_{\chi}\}$ and $\varnothing$ is the empty set. This can effectively be bypassed by setting $\Theta_\chi$ = 0.

\pdfbookmark{Note}{n}\section*{Note}

It is recommended that if \mcode{norm_flag} = 1 then $\tau \gg 1$. As can be seen above, as $\tau\rightarrow1$ then $\tf{a}_{\{L,R\}}(\CI,m,n-1)\rightarrow0$, and hence $\norm{\tf{c}}(\CI,m,n)\rightarrow1$.

\end{document}